
\subsection{Radiative Hydrodynamic Shock Simulation}

Gasses subjected to a strong adiabatic shock will be heated very strongly. If the level of heating is sufficient,
the shocked fluid may be expected to radiate.

This creates an extended `shock' region consisting of the adiabatic jump, followed by a radiating region with
increasing density. Eventually a cutoff is reached, either physically (optical opacity) or numerically (temperature floor
in radiation function).

In the limit where the radiation time is taken to be extremely short compared to the flow time, the familiar
adiabatic Rankine-Hugoniot condition that $E_2 = E_1$ is replaced by the isothermal condition $T_2 = T_1$, resulting
in an isothermal shock.

\subsubsection{Initial Conditions}

The radiating shock simulation divides the X component of a 1-, 2- or 3-dimensional grid up into three
sections, the preshock, cooling, and cold gas layers.

Imogen accepts as input the parameters \begin{tt}fractionPreshock\end{tt} and \begin{tt}fractionCold\end{tt}.
The remainder forms the cooling region; Imogen chooses the grid spacing such that the distance from the shock
to the cold layer matches the length returned by the \begin{tt}RadiatingFlowSolver\end{tt} class.

This distance is approximately equal to the cooling length $L_{\text{cool}} \approx v_x \frac{\dot{P}}{P}$ which
in turn is defined by the shock's Mach number.

Imogen chooses frames such that the shock is initially stationary.

The \begin{tt}RadiatingFlowSolver\end{tt} class will generate a cooling region solution that is effectively exact, far
more accurate than the CFD code's truncation error.

Imogen requires an artificial parameter \begin{tt}Tcutoff\end{tt} which is the temperature that the cooling flow is allowed
to reach, in units of the preshock temperature. This value must be at LEAST one. In practice, it should be slightly
larger ($1.02-1.05$) to make both the shock jump discontinuity and the cold cutoff $\mathbb{C}^0$ points better behaved.

Without a temperature cutoff, a hydrodynamic radiating flow with a reasonable radiation law will cool to zero temperature,
infinite density and zero velocity in finite time and distance.

\subsubsection{Analysis}

...
