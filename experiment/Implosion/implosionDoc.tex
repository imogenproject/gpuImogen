
\subsection{Implosion Symmetry Test}

The implosion symmetry test begins with initial conditions which have mirror symmetry
about the $x=y$ axis. Fluid mechanics dictates that then the exact solution will 
hold this symmetry forever.

Dimension-split codes are virtually never able to reproduce this condition forever
because their error commutators inevitably contain asymmetry-generating terms.

Because Imogen is dimensionally split, its error in this test is tracked by comparing
the ratio of the antisymmetric to symmetric components of the density.

\subsubsection{Initial Conditions}

Nominally any condition of mirror symmetry will do. The condition used by Athena consists of
a square domain of size 1x1, with
\begin{itemize}
\item $\rho(x+y < 1) = \rho_a$
\item $\rho(x+y \ge 1) = 1$
\item $P(x+y < 1) = P_a$
\item $P(x+y \ge 1) = 1$
\item $\vec{v} = 0$
\end{itemize}
By default the gas gamma is set to $7/5$ with $\rho_a = .125$ and $P_a = .14$.
All boundary conditions are mirrored.

The Imogen parameters are \begin{tt}Mcorner\end{tt} and \begin{tt}Pcorner\end{tt}.

This results in a standard Sod-like behavior in the middle of the corner/bulk region,
and a `jet' being launched along both edges, where the mirror boundary collides the two jets.
These jets promptly develop shear-flow instabilities. When they meet in the corner, they 
expel a two-lobed vortex down the $x=y$ line.

This is typically the point where the asymmetry component becomes obvious as this vortex is
intensely unstable to $x=y$-asymmetric disturbances.

\subsubsection{Analysis}


