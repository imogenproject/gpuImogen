

\subsection{Sedov-Taylor Explosion Test}

The ST test measures the code's ability to capture high-Mach shocks and its ability to handle
them in multiple dimensions.

The initial conditions (analytically) for a Sedov-Taylor explosion consist of a spherically symmetric
fluid in N dimensions with
\begin{itemize}
\item $\rho(r) = \rho_0 r^-j$
\item $v(r) = 0$
\item $E_{\text{tot}}(r) = E / (\frac{4}{3} \pi \epsilon^3)$
\end{itemize}
In the standard classical ST case, $j=0$ and $N=3$ (producing a spherical explosion that models
the adiabatic phase of a nuclear explosion or supernova).


Details are in Sedov (1950), the best known of the original 3 ST papers. A modern derivation
which includes a walk-through explanation of the analysis and model code can be found in
Timmes 2000 \& Kamm \& Timmes (2002).

Under most conditions, the result is expected to maintain N-spherical symmetry.

KT also describe the existence of choices of physical ($j < N + 1$, resulting in finite mass within 
finite volume) solutions which have opposing pressure \& density gradients; These would be unstable 
to Rayleigh-Taylor induced convection.


